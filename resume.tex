
%  ______     __     __  _____
%  / ___\ \   / /    / / |  _ \ ___  ___ _   _ _ __ ___   ___
% | |    \ \ / /    / /  | |_) / _ \/ __| | | | '_ ` _ \ / _ \
% | |___  \ V /    / /   |  _ <  __/\__ \ |_| | | | | | |  __/
%  \____|  \_/    /_/    |_| \_\___||___/\__,_|_| |_| |_|\___|
%

% My personal resume...

\documentclass{cv}

\begin{document}

% My Real Name
\name{Amlesh Sivanantham}

% Contract Information
\section{Contact Information}
\contact{samlesh@gmail.com}{(408) 219-6474}{http://zamlz.org}

% My skill set
\section{Skills}
\skills{Python, C++, C, Java, JavaScript,
Scheme, \LaTeX, Bash, Markdown, Verilog}
{TensorFlow, OpenAI Gym, PyTorch, NumPy, SciPy, Matplotlib, OpenCV}
{Git, Vim, Linux}

% Education History
% If you using present in datelist, use \normalsize{Present}
\section{Education}
\datelist{08.2017 - 05.2019,
\textsc{University of Southern California}\\
\textbf{M.S. in Computer Science}\\
\textit{Concentration: Intelligent Robotics},
09.2013 - 06.2017,
{\textsc{University of California, Santa Cruz}}\\
\textbf{B.S. in Computer Engineering}\\
\textbf{B.S. in Computer Science}\\
Thesis: \textit{Detecting Anomalies in Time-Series Data using\\
Long Short-Term Memory Networks} - Advisor: Dr.~Patrick Mantey}

% Work Experience
\section{Work Experience}
\datelist{09.2017 - \normalsize{Present},
\textbf{Graduate Research Assistant}\\
\textit{University of Southern California -
Robotic Embedded Systems Laboratory}\\
Perform graduate research in Deep Reinforcement Learning and it's application
to Robotics. Working for a PhD student on research problems related to perception
and navigation with Deep Reinforcement Learning.,
09.2016 - 06.2017,
\textbf{Undergraduate Research Assistant}\\
\textit{{University of California, Santa Cruz} -
Jack Baskin School of Engineering}\\
Performed undergraduate research in Machine Learning and Deep Learning
for the Smart Energy Analytic Disaggregation System project for
Dr.~Ali Adabi to explore methods to analyze and identify time-series
data.
}

% Workshop Papers
%\section{Workshop Papers}

% Project Section
\section{Projects}
\begin{adjustwidth}{0.5cm}{0.5cm}

\projectsubsectionold{Quadcopter Reinforcement Learning Agent}
{USC R.E.S.L.}
{Working on building an Reinforcement Learning agent that is capable of
navigating quickly through a cluttered environment. Train the agent in
simulation, and transfer the learnt policy to a real-world quadcopter.
\textit{In Progress}}{}

\projectsubsection{Imagination Augmented Agents for Rubik's Cubes}
{Jeju DL Camp 2018, S. Korea}
{Participated in the Jeju Deep Learning Camp where I worked to
implement the paper \textit{Imagination Augmented Agents for
Deep Reinforcement Learning} on a self-made Rubik's Cube environment}
{https://github.com/zamlz/dlcampjeju2018-I2A-cube}

\projectsubsectionold{Anomaly Detection in Time Series Data}
{Undergraduate Senior Thesis}
{Researched Deep Learning and implemented a long short-term memory
network that identifies if a given subsequence of a particular
time-series system is anomalous or not. The dataset that I worked with
was provided by my faculty advisor which corresponds to the energy usage
of an electric meter on the circuit that provides power to the water
pump. After training, the network was able to identify anomalies with
an accuracy of 90\%.}{}

\projectsubsectionold{Who's Lazy? Not Eye}
{Hack UCSC 2017}
{A vision therapy program for people with lazy eye using a standard
webcam. The app uses the webcam to constantly analyze the user's eyes
and notifies them when their eyes drift away. Particularly, the client
can pause any media application that the user has playing in the
background and will only let them resume their application once they
have focused their eyes. I worked on the algorithm that located the
position of the pupils using machine learning and identified whether
the pupils correlated with lazy eyes or not based on that position.}
{https://github.com/zamlz/hackucsc-lazyeye}

\projectsubsectionold{Othello (Reversi) for the PSoC 5LP}
{Project for Microcontroller System Design Class}
{Implemented Othello using C and the Cypress API for the Othello PSoC 5LP.
It was built using all the different concepts learned in class. Refer to the
report in the documentation directory in the repository for more details.}
{https://github.com/zamlz/Othello-PSoC-5LP}

\projectsubsectionold{UCSC Plaza}
{Project for Software Engineering Class}
{UCSC Plaza is an event manager designed for members of UCSC. This is simply
a prototype. Worked with JavaScript and JQuery to make the interface which
connects the front-end to the web-server. Project was developed using Scrum
practices.}
{https://github.com/Skm1221/CMPS115TeamProject}

\projectsubsectionold{HummusLite}
{Project on Logic Design}
{A simple project I worked on over Summer of 2016. Goal was to build a simple
CPU in Minecraft. The CPU supports 16 instructions and has a program data
size of 256 bytes. Each instruction is a byte. Programming for the CPU
requires using a punch board system that was also built from scratch in
Minecraft.}
{https://github.com/zamlz/hummuslite}
\end{adjustwidth}

% A more personal record about me
\section{Research Interests and Hobbies}
\begin{adjustwidth}{0.5cm}{0.5cm}
Enjoy researching about Reinforcement Learning and its applications
in Robotics and Competitive Games. Also a Linux enthusiast
currently learning/using Gentoo Linux. Amateur Competitive Super Smash
Bros. Melee Player. Also enjoy playing the violin and speedcubing.

\end{adjustwidth}

\end{document}

