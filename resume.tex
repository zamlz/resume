
%  ______     __     __  _____
%  / ___\ \   / /    / / |  _ \ ___  ___ _   _ _ __ ___   ___
% | |    \ \ / /    / /  | |_) / _ \/ __| | | | '_ ` _ \ / _ \
% | |___  \ V /    / /   |  _ <  __/\__ \ |_| | | | | | |  __/
%  \____|  \_/    /_/    |_| \_\___||___/\__,_|_| |_| |_|\___|
%

% My personal resume...

\documentclass{cv}

\begin{document}

% Contact information
\name{Amlesh Sivanantham}
\contact{zamlz@pm.me}{(408) 219-6474}
{zamlz.org}{http://zamlz.org}
{zamlz}{https://github.com/zamlz}

% Education History
% If you using present in datelist, use \normalsize{Present}
\section{Education}
%\imgsection{Education}{education}
\datelist{08.2017 - 05.2019,
\textsc{University of Southern California}\\
\textbf{M.S. in Computer Science (GPA: 3.78)}\\
\textit{Concentration: Intelligent Robotics},
09.2013 - 06.2017,
{\textsc{University of California, Santa Cruz}}\\
\textbf{B.S. in Computer Engineering (Honors)}\\
\textbf{B.S. in Computer Science}\\
Thesis: \textit{Detecting Anomalies in Time-Series Data using\\
Long Short-Term Memory Networks} - Advisor: Dr.~Patrick Mantey}

% My skill set
\section{Skills}
%\imgsection{Skills}{skills}
\skills{\textbf{Python}, \textbf{C++}, C, Java, JavaScript,
Scheme, \LaTeX, Bash, Verilog}
{\textbf{TensorFlow}, PyTorch, OpenAI Gym, PyTorch, NumPy, SciPy, Matplotlib}
%{Git, Linux \{Gentoo, Void, Ubuntu\}}

% Work Experience
\section{Work Experience}
%\imgsection{Work Experience}{work}
\datelist{09.2017 - \normalsize{Present},
\textbf{Graduate Research Assistant}\\
\textit{University of Southern California - Robotic Embedded Systems Laboratory}\\
Perform graduate research in Deep Reinforcement Learning and it's application
to Robotics. Some of the areas I have worked on have been to learn to infer
inverse dynamics of a system (system identification) and learning to integrate
control theory with current deep reinforcement learning algorithms. I also
participate in reading groups to be caught up with the current literature.}

\datelist{09.2016 - 06.2017,
\textbf{Undergraduate Research Assistant}\\
\textit{{University of California, Santa Cruz} - Jack Baskin School of Engineering}\\
Performed undergraduate research in Machine Learning and Deep Learning for the
Smart Energy Analytic Disaggregation System project for Dr.~Ali Adabi to
explore methods to analyze and identify time-series data.}

\datelistold{04.2017 - 06.2017,
\textbf{Academic Coursework Grader}\\
\textit{{University of California, Santa Cruz} - Jack Baskin School of Engineering}\\
Graded homework for Computational Models (CMPS 130) and Analysis of
Algorithms (CMPS 102).
}

% Workshop Papers
\section{Publications}
%\imgsection{Publications}{paper}
\begin{adjustwidth}{0.5cm}{0.5cm}

\paperlist{
    \textbf{W1.}, \bibentry{amlesh2018marlo}
}
\vspace{0.25cm}
\end{adjustwidth}

% Project Section
\section{Projects}
%\imgsection{Projects}{projects}
\begin{adjustwidth}{0.5cm}{0.5cm}

\projectsubsection{Learning Inverse Dynamics of a System for Deep RL}
{USC RESL}
{Instead of having a RL policy learn a mapping from states to actions, we had
it learn a mapping from states to desired states. We also learnt a inverse
dynamics model concerrently from data generated by the policy. We found that
the policy's performance was marginally worse than the standard approach.}{}

\projectsubsection{PPO with Curriculum Learning for Quadrotor Navigation}
{USC RESL}
{Used Proximal Policy Optimization (PPO) with curriculum learning to train a
policy to learning to fly a quadrotor in a simple OpenGL quadrotor simulator we
wrote. We we able to solve the task when we used perfect state information, but
when we changed the state to RGB image data from a camera and IMU information,
it was unable to learn the task.}{}

\projectsubsection{Imagination Augmented Agents for Rubik's Cubes}
{Jeju DL Camp 2018, S. Korea}
{Participated in the Jeju Deep Learning Camp 2018 where I worked to
reimplement the paper \textit{Imagination Augmented Agents for
Deep Reinforcement Learning} and adapt it to work for a Rubik's Cube OpenAI
Gym environment that I wrote.}
{https://github.com/zamlz/dlcampjeju2018-I2A-cube}


\projectsubsectionold{Deep Q-Learning with Structure2Vec for Solving the
Vehicle Routing Problem}{USC - Topics in Discrete Optimization and Learning}
{Worked to adapt previous work done by \textit{Hanjun et al.} that utilized
Q-learning and the Structure2Vec graph embedding to solve discrete optimization
problems. We adapted the approach to work with the domain of the Vehicle
Routing Problem.}
{https://github.com/zamlz/graph_comb_opt}

\projectsubsectionold{Anomaly Detection in Time Series Data}
{Undergraduate Senior Thesis}
{Researched Deep Learning and implemented a long short-term memory
network that identifies if a given subsequence of a particular
time-series system is anomalous or not. The dataset that I worked with
was provided by my faculty advisor which corresponds to the energy usage
of an electric meter on the circuit that provides power to the water
pump. After training, the network was able to identify anomalies with
an accuracy of 90\%.}{}

\projectsubsectionold{Who's Lazy? Not Eye}
{Hack UCSC 2017}
{A vision therapy program for people with lazy eye using a standard
webcam. The app uses the webcam to constantly analyze the user's eyes
and notifies them when their eyes drift away. Particularly, the client
can pause any media application that the user has playing in the
background and will only let them resume their application once they
have focused their eyes. I worked on the algorithm that located the
position of the pupils using machine learning and identified whether
the pupils correlated with lazy eyes or not based on that position.}
{https://github.com/zamlz/hackucsc-lazyeye}

\projectsubsectionold{Othello (Reversi) for the PSoC 5LP}
{Project for Microcontroller System Design Class}
{Implemented Othello using C and the Cypress API for the Othello PSoC 5LP.
It was built using all the different concepts learned in class. Refer to the
report in the documentation directory in the repository for more details.}
{https://github.com/zamlz/Othello-PSoC-5LP}

\projectsubsectionold{UCSC Plaza}
{Project for Software Engineering Class}
{UCSC Plaza is an event manager designed for members of UCSC. This is simply
a prototype. Worked with JavaScript and JQuery to make the interface which
connects the front-end to the web-server. Project was developed using Scrum
practices.}
{https://github.com/Skm1221/CMPS115TeamProject}

\projectsubsectionold{HummusLite}
{Project on Logic Design}
{A simple project I worked on over Summer of 2016. Goal was to build a simple
CPU in Minecraft. The CPU supports 16 instructions and has a program data
size of 256 bytes. Each instruction is a byte. Programming for the CPU
requires using a punch board system that was also built from scratch in
Minecraft.}
{https://github.com/zamlz/hummuslite}
\end{adjustwidth}

% A more personal record about me
\ifthenelse{\equal{\fullresume}{true}}{
\section{Interests and Hobbies}
%\imgsection{Interests and Hobbies}{hobbies}
\begin{adjustwidth}{0.5cm}{0.5cm}

I enjoy researching about all sorts of sciences in my free time, especially
Physics. I am also a Linux enthusiast who likes to use a variety of different
linux distros. Currently using both Gentoo and Void right now. I am also
working towards becoming a competitive Smash Bros. Melee player. I also actively
practice the violin and enjoy composing music on occasion. I also enjoy
speed solving a variety of puzzles (Rubik's Cubes, Megaminx, etc.).

\end{adjustwidth}
}{}

\bibliographystyle{abbrv}
\nobibliography{papers.bib}
\end{document}

